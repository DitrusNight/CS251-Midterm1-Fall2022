\documentclass{article}
\usepackage[utf8]{inputenc}
\usepackage{amssymb}

\title{CS251 Midterm 1 - Fall 2022}
\author{schari}
\date{September 2022}

\begin{document}

\maketitle

\section{Summations and Logarithm Rules}
\begin{itemize}
    \item Summations
        \begin{itemize}
            \item Given $c$ is a constant, $\sum_{i = m}^{n} c = c(n - m + 1)$
            \item $\sum_{i = 1}^{n} i = \frac{1}{2}n(n + 1)$
            \item $\sum_{i = 1}^{n} i^2 = \frac{1}{6}n(n + 1)(2n + 1)$
            \item Given a function $f(i)$, $\sum_{i = m}^{n} f(i) = \sum_{i = 1}^{n} f(i) - \sum_{i = 1}^{m - 1} f(i)$
        \end{itemize}
    \item Log Rules
        \begin{itemize}
            \item In CS 251, if you are just given a $\log(n)$ without a base, they probably mean $\log_2(n)$
            \item $\log(ab) = \log(a) + \log(b)$
            \item $\log(\frac{a}{b}) = \log(a) - \log(b)$
            \item Given 2 numbers $a$ and $b$, $\log_a(n) = \frac{\log_b(n)}{\log_b(a)}$
            \item $\log(n^a) = a \log(n)$
            \item $a^{\log_a(n)} = n$
            \item $a^{b \log_a(n)} = n^b$
        \end{itemize}
\end{itemize}

\section{Experimental Analysis}

\begin{itemize}
    \item Limitations
        \begin{itemize}
            \item Different machines can vary the run time % Make these more concrete phrases instead of from the slides
            \item other processes/noise
            \item May not be precise all the time 
        \end{itemize}
\end{itemize}

\section{Recursive Functions}
\begin{itemize}
    \item Functions that call themselves in order to solve simpler problems
    \item Recursive functions don't call themselves infinitely; eventually stop when they reach a base case
    \item 
\end{itemize}

\section{Runtime Analysis}
\begin{itemize}
    \item Represents the efficiency of an algorithm
    \item Three types of asymptotic runtime analysis: $O(n)$, $\Omega(n)$, and $\Theta(n)$
    \item $O(n)$
        \begin{itemize}
            \item The asymptotic upper bound
            \item Definition: Given functions $f(n)$ and $g(n)$, then $f(n) \in O(g(n))$ if there exists constants $c$ and $n_0$ where $0 \leq f(n) \leq cg(n)$ for all $n \geq n_0$
            \item In other words, $f(n) \in O(g(n))$ if $f(n)$ doesn't grow faster than $g(n)$.
            \item Growth order:
                \begin{itemize}
                    \item $O(1) < O(\log(n)) < O(n) < O(n \log(n)) < O(n^2) < O(n^3) < O(2^n) < O(n!)$
                \end{itemize}
            \item Going from the definition above, multiple functions can be big-$O$ of another function.
                \begin{itemize}
                    \item Ex: $n \in O(n)$, and $n \in O(n^2)$
                \end{itemize}
            \item Generally, if they're asking for big-$O$ of a function, you want to give the tightest bound of the function.
            \item When giving the $O(n)$ of a function, you take the fastest-growing term and remove the constants from it
                \begin{itemize}
                    \item Ex: $4n^2 + 2n\log(n) + 3n + 123456 \in O(n^2)$
                \end{itemize}
        \end{itemize}
    \item $\Omega(n)$
        \begin{itemize}
            \item Asymptotic lower bound
            \item Definition: Given functions $f(n)$ and $g(n)$, then $f(n) \in \Omega(g(n))$ if there exists constants $c$ and $n_0$ where $0 \leq cg(n) \leq f(n)$ for all $n \geq n_0$
            \item In other words, $f(n) \in \Omega(g(n))$ if $f(n)$ doesn't grow slower than $g(n)$.
            \item Like how multiple functions can be $O(n)$ of a function, multiple functions can also be $\Omega(n)$ of a function
                \begin{itemize}
                    \item Ex: $n \in \Omega(n)$, and $n \in \Omega(\log(n))$
                \end{itemize}
            \item Again, generally you should give the tightest bound
            \item Process for getting big-$\Omega$ of a function is same as getting big-$O$
        \end{itemize}
    \item $\Theta(n)$
        \begin{itemize}
            \item Asymptotic tightest bound of a function
            \item $f(n) \in \Theta(g(n))$ if $f(n)$ doesn't grow faster or slower than $g(n)$
        \end{itemize}
    \item Asymptotic Growth Properties
        \begin{itemize}
            \item If $f(n) \in O(g(n))$ and $f(n) \in \Omega(g(n))$, then $f(n) \in \Omega(g(n))$ and vice versa
            \item If $f(n) \in O(g(n))$ and $g(n) \in O(h(n))$, then $f(n) \in O(h(n))$
            \item If $f(n) \in \Omega(g(n))$ and $g(n) \in \Omega(h(n))$, then $f(n) \in \Omega(h(n))$
            \item If $f(n) \in \Theta(g(n))$ and $g(n) \in \Theta(h(n))$, then $f(n) \in \Theta(h(n))$
            \item If $f(n) \in O(h(n))$ and $g(n) \in O(h(n))$, then $f(n) + g(n) \in O(h(n))$
            \item If $f(n) \in \Theta(g(n))$, then $f(n) + g(n) \in \Theta(g(n))$
            \item If $f(n) \in O(g(n))$, then $g(n) \in \Omega(f(n))$
        \end{itemize}
\end{itemize}

\section{Arrays and LinkedLists}

\section{Stacks}
\begin{itemize}
    \item Data structure to store and remove data
    \item Last data pushed into the stack would be the first data popped off (LIFO)
        \begin{itemize}
            \item Think of it like a stack of plates; the last plate placed on top is the first plate taken from the stack
        \end{itemize}
    \item Standard methods for stacks:
        \begin{itemize}
            \item \verb|push()| - Add an element to the top of the stack
            \item \verb|pop()| - Remove the element from the top of the stack
            \item \verb|isEmpty()| - Whether or not there are elements on the stock
            \item \verb|size()| - Number of elements on the stack
            \item \verb|peek()| - View the element at the top of the stack without removing it
        \end{itemize}
    \item Implementation using Arrays vs LinkedLists
        \begin{itemize}
            \item Arrays: Lower memory overhead; unable to resize to accomodate more elements
            \item LinkedLists: Pointers require more memory; can expand to increase number of elements in the stack
        \end{itemize}
\end{itemize}

\section{Queues}
\begin{itemize}
    \item Data structure to store and remove data
    \item First data enqueued would be the first data dequeued (FIFO)
        \begin{itemize}
            \item Think of it as a queue of people; first person to enter is also the first person who gets served
        \end{itemize}
    \item Standard methods for stacks:
        \begin{itemize}
            \item \verb|enqueue()| - Add an element to the end of the queue
            \item \verb|dequeue()| - Remove the element from the front of the queue
            \item \verb|isEmpty()| - Whether or not there are elements in the queue
            \item \verb|size()| - Number of elements in the queue
            \item \verb|peek()| - View the element at the front of the queue without removing it
        \end{itemize}
    \item Implementation using Arrays vs LinkedLists
        \begin{itemize}
            \item Arrays: In addition to lower memory overhead, a queue implementation will loop from the end of array to the front, and will keep track of the index of the front and back of the queue (see slides for example).
            \item LinkedLists: Because LinkedLists are relatively resizable, you don't need to keep track of indices, but you will have to keep track of the front and back of the queue nodes.
        \end{itemize}
\end{itemize}

\section{Trees}

\end{document}

